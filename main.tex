% Sablon pentru realizarea lucrarii de licenta, conform cu recomandarile
% din ghidul de redactare:
% - https://fmi.unibuc.ro/finalizare-studii/
% - https://drive.google.com/file/d/1xj9kZZgTkcKMJkMLRuoYRgLQ1O8CX0mv/view

% Multumiri lui Gabriel Majeri, acest sablon a fost creat pe baza
% codului sursa a lucrarii sale de licenta. 
% Codul sursa: https://github.com/GabrielMajeri/bachelors-thesis
% Website: https://www.gabrielmajeri.ro/
%
% Aceast sablon este licentiat sub Creative Commons Attribution 4.0 International License.

\documentclass[12pt, a4paper]{report}

% Suport pentru diacritice și alte  simboluri
\usepackage{fontspec}
\usepackage{soul}
% Suport pentru mai multe limbi
\usepackage{polyglossia}

% Setează limba textului la română
\setdefaultlanguage{romanian}
% Am nevoie de engleză pentru rezumat
\setotherlanguages{english}

% Indentează și primul paragraf al fiecărei noi secțiuni
\SetLanguageKeys{romanian}{indentfirst=true}

% Suport pentru diferite stiluri de ghilimele
\usepackage{csquotes}

\DeclareQuoteStyle{romanian}
  {\quotedblbase}
  {\textquotedblright}
  {\guillemotleft}
  {\guillemotright}

% Utilizează biblatex pentru referințe bibliografice
\usepackage[
    style=numeric,
    maxbibnames=50,
    sorting=nty
]{biblatex}

\addbibresource{bibliography.bib}
% Setează spațiere inter-linie la 1.5
\usepackage{setspace}
\onehalfspacing

% Modificarea geometriei paginii
\usepackage{geometry}

% Include funcțiile de grafică
\usepackage{graphicx}
% Încarcă imaginile din directorul `images`
\graphicspath{{./images/}}

% Listări de cod
\usepackage{listings}

% Linkuri interactive în PDF
\usepackage[
    colorlinks,
    linkcolor={black},
    menucolor={black},
    citecolor={black},
    urlcolor={blue}
]{hyperref}

% Simboluri matematice codificate Unicode
\usepackage[warnings-off={mathtools-colon,mathtools-overbracket}]{unicode-math}

% Comenzi matematice
\usepackage{amsmath}
\usepackage{mathtools}

% Formule matematice
\newcommand{\bigO}[1]{\symcal{O}\left(#1\right)}
\DeclarePairedDelimiter\abs{\lvert}{\rvert}

% Suport pentru rezumat în două limbi
% Bazat pe https://tex.stackexchange.com/a/70818
\newenvironment{abstractpage}
  {\cleardoublepage\vspace*{\fill}\thispagestyle{empty}}
  {\vfill\cleardoublepage}
\renewenvironment{abstract}[1]
  {\bigskip\selectlanguage{#1}%
   \begin{center}\bfseries\abstractname\end{center}}
  {\par\bigskip}

% Suport pentru anexe
\usepackage{appendix}

% Stiluri diferite de headere și footere
\usepackage{fancyhdr}

% Metadate
\title{Corectarea automată cu ajutorul LLM-urilor : Studiu de caz - Bacalaureat Informatică}
\author{Firca Liviu Nicolae}

% Generează variabilele cu @
\makeatletter

\begin{document}

% Front matter
\cleardoublepage
\let\ps@plain

% Pagina de titlu
\include{0-title}
\restoregeometry
\newgeometry{
    margin=2.5cm
}

\fancypagestyle{main}{
  \fancyhf{}
  \renewcommand\headrulewidth{0pt}
  \fancyhead[C]{}
  \fancyfoot[C]{\thepage}
}

\addtocounter{page}{1}

% Rezumatul
\begin{abstractpage}

\begin{abstract}{romanian}
Acest studiu se întreabă dacă examenul de bacalaureat în informatică poate să fie corectat automat folosind Large Language Models (LLM).
Motivul acestei interogări este pentru că bacalaureatul este un examen foarte important, și LLM-urile ar putea fi mai obiective și consistente
decât corectorii umani. De asemenea foarte mult timp este investit în corectarea acestor lucrări. În plus studiul a explorat
dacă LLM-urile pot genera feedback automat pentru profesori, pe baza corectării LLM-urilor, pentru a sprijini îmbunătățirea predării.

Pentru a determina dacă bacalaureatul poate fi corectat folosind LLM-uri, am măsurat eroarea notelor atribuite de LLM folosind Mean Absolute Error (MAE),
și pe baza acestei evaluări a performanței am concluzionat că este posibil să corectezi lucrările de bacalaureat la informatică folosind LLM-uri.
Pentru a evalua calitatea feedbackului, am verificat dacă acesta identifică toate greșelile comune ale studenților și dacă este corect din punct de vedere al conținutului.
\end{abstract}

\begin{abstract}{english}
This study investigates whether the computer science baccalaureate exam can be graded automatically using Large Language Models (LLMs).
The motivation for this is that the baccalaureate is a very important exam, and LLMs could potentially be more objective and consistent than human graders. Additionally, a significant amount of time is invested in grading these papers.
Furthermore, this study explores if LLMs can correctly generate feedback for teachers, on the basis of the grading provided by LLMs, in order to 
enhance their teaching.


To determine whether the baccalaureate can be graded using LLMs, the error of the grades assigned by the LLM was measured using Mean Absolute Error (MAE),
 and based on this performance evaluation, it shows that LLMs can be effectively used to grade computer science responses on the baccalaureate exam.
In order to evaluate the quality of the feedback, we checked if it contained all the common errors made by the students and if the 
feedback actually contained any mistakes that the students didn't make.

\end{abstract}

\end{abstractpage}

\tableofcontents

% Main matter
\cleardoublepage
\pagestyle{main}
\let\ps@plain\ps@main

\chapter{Introducere}
Bacalaureatul reprezintă unul dintre cele mai importante examene din sistemul de învățământ, oferind accesul la studiile universitare.
Admiterea în anumite instituții de învățământ superior, precum și acordarea burselor în primul an de studiu, se pot baza pe rezultatele obținute la examenul de Bacalaureat, demonstrând astfel importanța acestora.
Din acest motiv, obiectivitatea și acuratețea în corectare sunt necesare.
În plus de acest lucru, profesorii investesc foarte mult timp pentru a corecta lucrările de bac. Aproximativ 100.000 de elevi dau bacul în fiecare an, și corectarea unei lucrări în general este lentă.

O tehnologie foarte promițătoare care ar putea a priori rezolva această problemă este Large Language Models (prescurtat LLM). Ele sunt o funcție matematică cu scopul de a procesa limbajul (în majoritate textul) pentru a obține obiectivul dorit (o reprezentare vectorială a propoziției, un răspuns adecvat la o întrebare etc.).
Numele lor vine din faptul că numărul de parametri pe care îl are un LLM și cantitatea textului la care are acces la antrenare sunt factorii cei mai importanți în performanța sa.
LLM-urile au fost folosite înainte de acest studiu pentru a nota răspunsurile în cadrul Massive Open Online Courses (MOOC) — cursuri online disponibile publicului, accesibile unui număr mare de participanți prin internet \cite{golchin}.
Rezultatele au fost promițătoare, mai ales atunci când a existat un barem explicit pentru notare. Ele s-au arătat a fi mai consistente (erau mai apropiate de notele asignate de expert) decât corectorii umani \cite{golchin}.

Pentru că nu există date publice cu lucrări de bac, cât și notele lor primite, ne reducem abordarea doar la bacalaureatul de informatică pentru a ușura colectarea datelor, dar și studiul în general.
În plus de acest lucru, baremul acestui examen este destul de obiectiv.
Studiul de față își propune să investigheze două direcții de cercetare:
Este posibil de a corecta lucrările de bacalaureat în informatică folosind un LLM? și
Este posibil de a da feedback unui profesor folosind un LLM în acest context?
Prima oară, o să explicăm noțiunile și resursele folosite pentru studiu în preliminarii. După aceea, o să explorăm modul în care datele au fost create, dar și prelucrate, în capitolul 3.
În capitolul 4 va fi explicată corectarea automată a lucrărilor produse de studenți, dar și feedbackul generat pentru profesor
(dacă elevii ar face parte din aceeași clasă, ce informații i-ar putea fi oferite profesorului pentru a îmbunătăți predarea sa?).
În cele din urmă, o să expunem rezultatele studiului în concluzie.
\chapter{Preliminarii}

O sa explicam tehnologiile folosite dar si anumite notiuni utilizate in acest studiu.

Cel mai invechit LLM folosit este GPT 4 \cite{gpt4}. Modelul a fost publicat in Martie 2023, si faptul ca este cel mai vechi va fi important ulterior.

Al doilea LLM "invechit" este DeepSeek-V3, care a fost publicat in Decembrie 2024 \cite{v3}. A introdus diferite mecanisme care i-au permis sa fie mai ieftin decat 
competitia sa, dar tot pastrand o performanta similara.

Al treilea LLM studiat este DeepSeek-R1, care a fost publicat in Ianuarie 2025 \cite{r1} . In contrast cu GPT 4, Decembrie-V3 si GPT 4.1, acesta este un thinking model, 
adica LLM-ul este fortat sa "gandeasca" inainte sa dea un raspuns utilizatorului. Partea de gandire consta in incercarea a mai multor abordari, dar si
un spirit mai critic, care este mai greu de convins. Tipic vorbind LLM-ul incearca sa separe problema in mai multe etape de rezolvare in partea de gandire.
Noutatea adusa de acest LLM este ca e mult mai ieftin decat competitia lui, dar si mai performant (decat LLM-urile de gandire care au fost publicate inaintea lui)

Gemini 2.5 este, al doilea cel mai nou LLM utilizat in acest studiu \cite{gemini} . Nu sunt foarte multe cunoscute despre el, in afara ca si el este 
un thinking model, si ca a ridicat performanta state-of-the-art la mai multe capacitati, inclusiv matematica \cite{gemini}. Exista doua variante: cea pro si cea flash, 
evident varianta pro este cea mai scumpa.

Cel mai recent LLM studiat este GPT 4.1 \cite{gpt41}. Modelul a fost publicat in Aprilie 2025, si are o performanta mult mai buna decat predecesorul sau, GPT 4o.

Acuma ca am introdus tehnologiile folosite, trebuie si sa precizam notiunile folosite
\begin{itemize}
    \item Inferenta: in acest context, se refera la procesul prin care modelul generează un răspuns pornind de la o interogare data de la utilizator.
    Un lucru important de notat este ca s-a observat ca mai multe resurse alocate inferentei (mai multe cuvinte generate,
    LLM-ul este instruit sa detalieze mai mult rationamentul sau s.a.m.d. )
    rezulta in performanta mai buna. De exemplu s-a arata ca daca instruiesti un LLM sa se gandeasca pas cu pas, si nu sa raspunda pur si simplu, performanta creste \cite{cot}.
    \item Antrenarea : in acest context, se referla la procesul de a lua un model neinitializat, care nu stie nici macar sa vorbeasca, si sa
    il inbunatatesti pe baza anumitor date (text de pe internet, carti, s.a.m.d.). Aceasta inbunatatire se face schimband parametrii modelului.
    \item Fine-tuning : se refera la procesul de a lua un model deja antrenat, si de ai schimba parametrii a doua oara, pentru al face mai bun
    pentru o anumata sarcina. Lucrul acesta ar merge bine de exemplu daca luam un LLM care a fost antrenat sa faca "orice", si il specializam sa 
    raspunda la emailuri. Idea de fine-tuning este interesanta pentru acest studiu, pentru ca daca ai face un fine-tuning pe LLM-urile studiate, 
    ar fi foarte probabil ca performanta sa creasca putin.

\end{itemize}

O premisa \textbf{foarte importanta} a acestui studiu este ca daca distanta medie intre nota adevarata si nota asignata (mean absolute error, sau MAE) de LLM este sub 5 puncte din 100, 
atunci acuratetea este judecata ca fiind destul de buna pentru a corecta notele de bac. 
Am luat ca metrica de performanta MAE pentru ca a fost folosita inainte pentru a masura performanta LLM-urilor la notarea lucrarilor \cite{golchin}.
Este si mai intuitiv sa luam MAE in loc de MSE (mean squared error) in acest caz, pentru ca si erorile mici sunt importante, o diferenta 
mica intre asteptarile studentului si nota primita ar
putea face studentul sa depuna o contestatie. Motivul pentru care pragul pentru reusita este setat la o eroare de 5 puncte din 100 este pentru ca 
\begin{enumerate}
    \item Chiar daca baremul este destul de obiectiv, exista totusi o mica parte de interpretare subiectiva
    \item Pana in 2016, o nota trebuia sa se schimbe cu 5 puncte ca sa fie 
\end{enumerate}
\chapter{Colectarea și prelucrarea datelor}
\section{Colectarea datelor}
Datele au fost generate de studenții facultății. Modul în care au lucrat a fost următorul: li s-a asignat un subiect de bacalaureat la informatică din 2020 până în 2024 și au trebuit să își aleagă un număr aleatoriu dintre 5 și 10. După aceea au fost instruiți să facă greșeli astfel încât nota lor să corespundă cu numărul ales.  
Pentru a facilita corectarea lucrărilor, studenții au și trebuit să indice greșelile făcute, dar și nota pe care și-au asignat-o.  
Am primit în total 42 de lucrări, dintre care au fost filtrate 9 lucrări pentru că aveau date de proastă calitate.  
Problemele pe care le-am întâlnit aici erau multiple:  
\begin{itemize}
  \item Niște studenți pur și simplu nu au ales subiectul care trebuie (sesiunea greșită, profilul Științele Naturii în loc de Matematică-Informatică)
  \item Niște studenți au trimis în formatul greșit (fișier .docx a interferat cu sintaxa colului, fișiere Excel în loc de .txt)
  \item Niște studenți pur și simplu nu erau programatori puternici și nu s-au verificat destul, deci când a fost corectată lucrarea au
  avut mai multe greșeli decât au raportat. Acest lucru este problematic pentru că încetinește mult corectarea lucrărilor. Totuși, studenții erau din anul întâi și aceste greșeli au fost cel mai probabil neintenționate.
\end{itemize}

Pentru a evita aceste probleme puteam face 2 lucruri:
\begin{itemize}
  \item Să creăm un formular pentru submitere
  \item Să mobilizăm studenți cu mai multă experiență în programare
\end{itemize}


\section{Prelucrarea datelor}
Datele folosite au fost nu numai lucrările studenților, dar și subiectele și baremele fiecărui an. Acestea au trebuit să fie rescrise pentru a avea o structură mai adaptată pentru LLM-uri.  
În plus față de acestea, au fost rescrise într-un limbaj mai simplu și ușor de înțeles. Am filtrat a doua oară lucrările studenților până am avut 23 de lucrări în total. Acest lucru a fost făcut din două motive: a limita costurile lucrării (fiecare query costă relativ mult) și a permite testarea a mai multor LLM-uri, dar și pentru a avea date de calitate.  
Lucrările au fost recorectate, cu atenție sporită la barem, dar și la potențialele erori făcute de elev (tot conținutul lucrărilor a trebuit recorectat, pentru a garanta o notă cât mai apropiată de realitate).

\chapter{Corectarea automată și feedback automat}
\section{Corectarea automată}

Una dintre premisele lucrării este că, dacă LLM-ul are o eroare mai mică de 0.5 puncte din 10, atunci ar putea teoretic înlocui un corector uman.  
Ne întrebăm în primul rând dacă cea mai directă abordare (se dă toată lucrarea, tot baremul și tot subiectul și este instruit să noteze lucrarea pe baza baremului) produce rezultate bune.  
Pentru acest lucru o să folosim GPT-4.1 din două motive: nu este un thinking model (vrem să avem abordarea cât mai simplă) și este un model recent pentru studiu.  
Eroarea folosită este mean absolute error (MAE), dar se ia în calcul și cel mai prost rezultat — CMPR — când distanța între nota reală și cea asignată a fost cea mai mare.

\begin{table}[h!]
\centering
\begin{tabular}{|l|c|c|}
\hline
\textbf{Modelul} & \textbf{MAE} & \textbf{CMPR} \\
\hline
GPT 4.1 & 4.78 & 18\\
\hline
\end{tabular}
\caption{Prima abordare, punctaj acordat din 100 de puncte}

\end{table}
Conform datelor, modelul se încadrează tehnic în limita superioară impusă (5 puncte din 100). Totuși, este foarte aproape de această limită, și CMPR este foarte mare.

În plus, analizând justificarea acordată a notelor, puteam observa mai multe lucruri în neregulă:
\begin{itemize}
  \item Uneori nu corectează pe baza baremului. De exemplu, dacă vede un program scris „foarte leneș”, îi va da 0 puncte, chiar dacă baremul generos de la bac i-ar acorda 5 puncte.
  \item Uneori, LLM-ul a rescris codul cu intenția de a-l puncta (lucru care se întâmplă dacă textul dat este foarte lung), dar a omis o parte a codului pe care 
  a judecat-o irelevantă. După ce a trecut prin cerințele baremului, a depunctat studentul pentru că lipsește partea de cod pe care nu a vrut să o copieze.
  Cred că lucrul acesta are două cauze simultane: un LLM nu se poate „gândi” la tot textul în același timp, deci nu își dă seama de eroarea comisă. În plus, planificarea în avans a fost clar proastă.
  \item Uneori nu și-a dat seama că răspunsul corect se afla în barem, deci a încercat să găsească singur soluția, dar nu a fost corectă. Deci corectarea a fost și ea greșită.
\end{itemize}


Multe dintre neînțelegerile sale au fost din cauza contextului prea mare, care l-a făcut să ignore baremul sau să corecteze greșit.  
Din cauza asta, avem o a doua abordare: ne dăm seama că fiecare întrebare de la bac este independentă, deci se împarte fiecare lucrare în 7 „lucrări”, iar baremul și subiectul
sunt și ele împărțite cum trebuie. După ce LLM-ul corectează tot, sunt 
recombinate toate cele 7 diviziuni în aceeași lucrare, și numai pe baza lucrării întregi se măsoară performanța. În plus folosim mai multe modele, inclusiv și thinking models. Din curiozitate, am inclus și GPT-4, pentru a vedea cât de important este factorul că modelele sunt noi.  
Folosind aceleași metrici:


\begin{table}[h!]
\centering
\begin{tabular}{|l|c|c|}
\hline
\textbf{Modelul} & \textbf{MAE} & \textbf{CMPR} \\
\hline
GPT 4.1 & 3.63 & 12\\

\hline
GPT 4 & 11.06 & 25.5 \\
\hline
Gemini 2.5 flash & 3.67 & 11\\


\hline
Gemini 2.5 pro & 3.09 & 11\\
\hline
Deepseek v3  & 5.8 & 14 \\

\hline
Deepseek r1 & 4.78 & 14 
\\
\hline

\end{tabular}
\caption{A doua abordare, punctaj acordat din 100 de puncte}

\end{table}

Putem remarca imediat că GPT-4 este un model foarte prost în comparație cu ceilalți, deci confirmă ideea că LLM-urile s-au ameliorat foarte mult în anii recenți  
și nu mai este cazul să folosim unul care nu este recent. Un LLM de actualitate este mult mai performant și, de multe ori, mult mai ieftin de folosit.  
Un alt lucru care iese la iveală este că resursele alocate inferenței sunt cruciale: există o diferență semnificativă între Gemini 2.5 Flash și Gemini 2.5 Pro de 0.5 puncte.  
În plus față de asta, există o diferență semnificativă între DeepSeek v3 și DeepSeek R1 de 1 punct, susținând iarăși ideea că mai multe resurse alocate la inferență rezultă în predicții mai bune.  
Deci se pot distinge doi factori cruciali în a determina performanța predicțiilor:
\begin{itemize}
  \item Resursele alocate inferenței
  \item Cât de recent este LLM-ul folosit (cu cât este mai recent, cu atât mai bine)
\end{itemize}


Putem remarca și faptul că GPT-4.1 și Gemini 2.5 Pro au avut erori foarte mici: 3.6 și 3. Acest lucru este destul de impresionant, pentru că există o parte subiectivă  
în a interpreta baremul, iar rezultatul cel mai bun (3) este cu mult sub limita superioară impusă. CMPR este totuși destul de mare pentru ambele modele.  
Două greșeli pe care am putut să le identific sunt următoarele:
\begin{itemize}
  \item Un student a făcut o greșeală în logica codului, iar Gemini 2.5 Pro a spus că algoritmul este greșit, când era clar că era doar o mică greșeală de logică.
  \item Un student nu a respectat cerința, dar GPT-4.1 i-a acordat toate punctele, pentru că nu și-a dat seama că structurile \textbf{repetă ... până când ...} și \textbf{până când ... repetă ...}  
  sunt foarte diferite în contextul cerinței: trebuia ca studentul să înlocuiască a doua structură în pseudocod cu prima, pentru a arăta că a înțeles cum se pot echivala algoritmic ambele.  
  Studentul nu a folosit structura care trebuie, dar a primit totuși toate punctele.
\end{itemize}

Ca soluție la aceste probleme, ar putea fi folosit un barem mai explicit, mai redundant, sau poate chiar și fine-tuning.



\section{Feedback automat}
Vrem să ne interesăm dacă LLM-urile pot să analizeze cum trebuie greșelile studenților
și să propună feedback relevant profesorului (ce parte din materie nu a fost bine înțeleasă de către studenți și trebuie predată mai bine).
Pentru asta, luăm cele top 2 modele în performanță și le testăm. Datele primite ca input pentru LLM sunt corectările lucrărilor obținute, fără subiect sau lucrarea propriu-zisă (a trebuit să justifice nota LLM-ul, deci există destul context doar în corectare).
Această parte este mult mai subiectivă decât cea precedentă, pentru că este greu să cuantifici calitatea unui feedback.

Problemele comune ale lucrărilor, identificate de mine, sunt:
\begin{itemize}
\item sintaxa lui C++
\item Algoritmica nu este înțeleasă bine
\item Întrebările nu sunt citite bine și, de multe ori, studenții s-au aruncat în a scrie o soluție fără să se gândească
\item I/O este implementat prost
\item Prelucrarea stringurilor (char[ ]) nu este bine înțeleasă
\item Studenții pierd foarte multe puncte la întrebările cu răspunsuri multiple
\end{itemize}

Problemele comune, identificate de GPT 4.1 sunt:
\begin{enumerate}
\item Algoritmica nu este înțeleasă bine
\item Studenții nu lucrează bine cu matrice (acest feedback este categoric fals)
\item Prelucrarea stringurilor (char[ ]) nu este bine înțeleasă
\item I/O este implementat prost
\item Parametrii referință (int f(\&n)) nu sunt bine folosiți
\item Inițializările sunt prost făcute
\end{enumerate}

Deci LLM-ul a identificat niște probleme foarte importante, unele care nu au fost identificate de mine. Totuși (2) este un feedback greșit.
Ceea ce lipsește în acest feedback este că sintaxa nu a fost bine înțeleasă și că studenții pierd prea multe puncte la întrebările cu răspuns multiplu.

Când vine vorba de Gemini 2.5 Pro, a identificat toate problemele pe care le-am văzut (cât si GPT 4.1), mai puțin faptul că studenții au pierdut prea multe puncte la întrebările cu răspuns multiplu.
Nu am primit feedback greșit din partea lui Gemini 2.5 Pro.

Pentru a completa partea de feedback,
le-am cerut LLM-urilor să genereze 30 de probleme,
pentru ca studenții să se poată antrena cu ele.
Din cele 30 generate de GPT 4.1, 10 au fost irelevante, și dintre cele 30 generate de Gemini, 7 au fost irelevante.

Putem trage concluzia că:
\begin{itemize}
\item Se pare că o performanță mai bună în MAE și CMPR la corectare
se transpune într-o performanță mai bună când vine vorba de feedback
\item Gemini 2.5 Pro este cel mai bun model, și la corectare automată, dar și la feedback
\end{itemize}

Totuși, ambele LLM-uri au dat feedback relevant.
\chapter{Concluzii}
Pe baza rezultatelor din Capitolul 4, se poate trage concluzia că există o probabilitate ridicată ca lucrările de bacalaureat la informatică să poată fi corectate automat cu un LLM. De asemenea, este posibilă implementarea unui sistem automatizat de feedback pentru profesori.

Alte rezultate din acest studiu arată că factorii cei mai importanți pentru acuratețea corectării sunt:
\begin{itemize}
\item Cât de recent este LLM-ul
\item Resursele alocate pentru inferență (putere de calcul \times timp, cât de detaliată și „gândită” este corectarea LLM-ului)
\end{itemize}

În plus, implementarea unei notări automate trebuie să împartă lucrarea în părți cât mai mici posibil: dacă textul dat LLM-ului este prea mare, atunci LLM-ul poate omite baremul sau poate comite alte greșeli.

O ipoteză care reiese din studiu este că, dacă un LLM are un MAE și un CMPR mai bun la notarea lucrărilor, atunci este mai bun pentru a transmite feedback. Pentru a dezvolta și mai departe rezultatele studiului și pentru a fi cât se poate de siguri de rezultatele sale, am putea relua evaluarea performanței ca în Capitolul 4, dar cu un set de date mai mare (>1000 lucrări).

Studiul de față nu propune nicio schimbare socială, dar dacă LLM-urile ar prelua corectarea lucrărilor de bacalaureat, atunci am recomanda păstrarea corectării umane cel puțin pentru lucrările cu notă contestată.

\printbibliography[heading=bibintoc]

\end{document}