\begin{abstractpage}

\begin{abstract}{romanian}
Acest studiu se întreabă dacă examenul de bacalaureat în informatică poate să fie corectat automat folosind Large Language Models (LLM).
Motivul acestei interogări este pentru că bacalaureatul este un examen foarte important, și LLM-urile ar putea fi mai obiective și consistente
decât corectorii umani. De asemenea foarte mult timp este investit în corectarea acestor lucrări. În plus studiul a explorat
dacă LLM-urile pot genera feedback automat pentru profesori, pe baza corectării LLM-urilor, pentru a sprijini îmbunătățirea predării.

Pentru a determina dacă bacalaureatul poate fi corectat folosind LLM-uri, am măsurat eroarea notelor atribuite de LLM folosind Mean Absolute Error (MAE),
și pe baza acestei evaluări a performanței am concluzionat că este posibil să corectezi lucrările de bacalaureat la informatică folosind LLM-uri.
Pentru a evalua calitatea feedbackului, am verificat dacă acesta identifică toate greșelile comune ale studenților și dacă este corect din punct de vedere al conținutului.
\end{abstract}

\begin{abstract}{english}
This study investigates whether the computer science baccalaureate exam can be graded automatically using Large Language Models (LLMs).
The motivation for this is that the baccalaureate is a very important exam, and LLMs could potentially be more objective and consistent than human graders. Additionally, a significant amount of time is invested in grading these papers.
Furthermore, this study explores if LLMs can correctly generate feedback for teachers, on the basis of the grading provided by LLMs, in order to 
enhance their teaching.


To determine whether the baccalaureate can be graded using LLMs, the error of the grades assigned by the LLM was measured using Mean Absolute Error (MAE),
 and based on this performance evaluation, it shows that LLMs can be effectively used to grade computer science responses on the baccalaureate exam.
In order to evaluate the quality of the feedback, we checked if it contained all the common errors made by the students and if the 
feedback actually contained any mistakes that the students didn't make.

\end{abstract}

\end{abstractpage}