\chapter{Concluzii}
Pe baza rezultatelor din Capitolul 4, se poate trage concluzia că există o probabilitate ridicată ca lucrările de bacalaureat la informatică să poată fi corectate automat cu un LLM. De asemenea, este posibilă implementarea unui sistem automatizat de feedback pentru profesori.

Alte rezultate din acest studiu arată că factorii cei mai importanți pentru acuratețea corectării sunt:
\begin{itemize}
\item Cât de recent este LLM-ul
\item Resursele alocate pentru inferență (putere de calcul \times timp, cât de detaliată și „gândită” este corectarea LLM-ului)
\end{itemize}

În plus, implementarea unei notări automate trebuie să împartă lucrarea în părți cât mai mici posibil: dacă textul dat LLM-ului este prea mare, atunci LLM-ul poate omite baremul sau poate comite alte greșeli.

O ipoteză care reiese din studiu este că, dacă un LLM are un MAE și un CMPR mai bun la notarea lucrărilor, atunci este mai bun pentru a transmite feedback. Pentru a dezvolta și mai departe rezultatele studiului și pentru a fi cât se poate de siguri de rezultatele sale, am putea relua evaluarea performanței ca în Capitolul 4, dar cu un set de date mai mare (>1000 lucrări).

Studiul de față nu propune nicio schimbare socială, dar dacă LLM-urile ar prelua corectarea lucrărilor de bacalaureat, atunci am recomanda păstrarea corectării umane cel puțin pentru lucrările cu notă contestată.