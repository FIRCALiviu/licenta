\chapter{Introducere}
Bacalaureatul reprezintă unul dintre cele mai importante examene din sistemul de învățământ, oferind accesul la studiile universitare.
Admiterea în anumite instituții de învățământ superior, precum și acordarea burselor în primul an de studiu, se pot baza pe rezultatele obținute la examenul de Bacalaureat, demonstrând astfel importanța acestora.
Din acest motiv, obiectivitatea și acuratețea în corectare sunt necesare.
În plus de acest lucru, profesorii investesc foarte mult timp pentru a corecta lucrările de bac. Aproximativ 100.000 de elevi dau bacul în fiecare an, și corectarea unei lucrări în general este lentă.

O tehnologie foarte promițătoare care ar putea a priori rezolva această problemă este Large Language Models (prescurtat LLM). Ele sunt o funcție matematică cu scopul de a procesa limbajul (în majoritate textul) pentru a obține obiectivul dorit (o reprezentare vectorială a propoziției, un răspuns adecvat la o întrebare etc.).
Numele lor vine din faptul că numărul de parametri pe care îl are un LLM și cantitatea textului la care are acces la antrenare sunt factorii cei mai importanți în performanța sa.
LLM-urile au fost folosite înainte de acest studiu pentru a nota răspunsurile în cadrul Massive Open Online Courses (MOOC) — cursuri online disponibile publicului, accesibile unui număr mare de participanți prin internet \cite{golchin}.
Rezultatele au fost promițătoare, mai ales atunci când a existat un barem explicit pentru notare. Ele s-au arătat a fi mai consistente (erau mai apropiate de notele asignate de expert) decât corectorii umani \cite{golchin}.

Pentru că nu există date publice cu lucrări de bac, cât și notele lor primite, ne reducem abordarea doar la bacalaureatul de informatică pentru a ușura colectarea datelor, dar și studiul în general.
În plus de acest lucru, baremul acestui examen este destul de obiectiv.
Studiul de față își propune să investigheze două direcții de cercetare:
Este posibil de a corecta lucrările de bacalaureat în informatică folosind un LLM? și
Este posibil de a da feedback unui profesor folosind un LLM în acest context?
Prima oară, o să explicăm noțiunile și resursele folosite pentru studiu în preliminarii. După aceea, o să explorăm modul în care datele au fost create, dar și prelucrate, în capitolul 3.
În capitolul 4 va fi explicată corectarea automată a lucrărilor produse de studenți, dar și feedbackul generat pentru profesor
(dacă elevii ar face parte din aceeași clasă, ce informații i-ar putea fi oferite profesorului pentru a îmbunătăți predarea sa?).
În cele din urmă, o să expunem rezultatele studiului în concluzie.