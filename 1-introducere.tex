\chapter{Introducere}
Bacalaureatul reprezintă unul dintre cele mai importante examene din sistemul de învățământ, oferind accesul la studiile universitare.
Admiterea în anumite instituții de învățământ superior, precum și acordarea burselor în primul an de studiu, se pot baza pe rezultatele obținute la examenul de Bacalaureat, demonstrand astfel importanța acestora. Din acest motiv obiectivitatea si corectarea cat se poate de buna este necesara.
In plus de acest lucru, foarte mult timp este depus de profesori pentru a corecta lucrarile de bac. Aproximativ 100 000 de elevi dau bacul in fiecare an, si corectarea unei lucrari in general este lenta.

O tehnologie \cite{golchin} foarte promitatoare care ar putea a priori rezolva aceasta problema este Large Language Models (prescurtat LLM). Ele sunt folosite cu scopul de a procesa limbajul (in majoritate textul)
pentru a obtine rezultatul dorit (o reprezentare vectoriala a propozitiei, un raspuns adecvat la o intrebare s.a.m.d.). Numele lor vine din faptul ca numarul de parametrii care are un LLM si cantitatea textului la care are acces la antrenare este factorul cel mai important in performanta sa. Este deja stiut ca LLM-urile sunt mai consistente decat corectorii umani .

Pentru ca nu exista date publice cu lucrari de bac cat si notele lor primite, ne reducem abordarea doar la bacalaureatul de informatica pentru a usura colectarea datelor dar si studiul in general. Studiul de fata isi propune sa investigheze doua directii de cercetare : Este posibil de a corecta lucrarile de bacalaureat folosind un LLM? si Este posibil de a da feedback unui profesor folosind un LLM?