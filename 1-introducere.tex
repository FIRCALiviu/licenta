\chapter{Introducere}
Bacalaureatul reprezintă unul dintre cele mai importante examene din sistemul de învățământ, oferind accesul la studiile universitare.
Admiterea în anumite instituții de învățământ superior, precum și acordarea burselor în primul an de studiu, se pot baza pe rezultatele obținute la examenul de Bacalaureat, demonstrand astfel importanța acestora. 
Din acest motiv, obiectivitatea și acuratețea în corectare sunt necesare.
In plus de acest lucru, profesorii investesc foarte mult timp pentru a corecta lucrarile de bac. Aproximativ 100 000 de elevi dau bacul in fiecare an, si corectarea unei lucrari in general este lenta.

O tehnologie foarte promitatoare care ar putea a priori rezolva aceasta problema este Large Language Models (prescurtat LLM). Ele sunt o functie matematica cu scopul de a procesa limbajul (in majoritate textul)
pentru a obtine obiectivul dorit (o reprezentare vectoriala a propozitiei, un raspuns adecvat la o intrebare s.a.m.d.). Numele lor vine din faptul ca numarul de parametrii care are un LLM si cantitatea textului la care are acces la antrenare este factorul cel mai important in performanta sa.
LLM-urile au fost folosite inainte de acest studiu pentru  a nota răspunsurile în cadrul Massive Open Online Courses (MOOC) — cursuri online disponibile publicului, accesibile unui număr mare de participanți prin internet \cite{golchin}. Rezultatele au fost promițătoare, mai ales atunci când a existat un barem explicit pentru notare. 
Ele s-au aratat a fi mai consistente (variatia notelor atribuite era mai mica) decat corectorii umani \cite{golchin}.

Pentru ca nu exista date publice cu lucrari de bac cat si notele lor primite, ne reducem abordarea doar la bacalaureatul de informatica pentru a usura colectarea datelor dar si studiul in general.
In plus de acest lucru baremul acestui examen este destul de obiectiv.
Studiul de fata isi propune sa investigheze doua directii de cercetare : Este posibil de a corecta lucrarile de bacalaureat in informatica folosind un LLM? si Este posibil de a da feedback unui profesor folosind un LLM in acest context ?
Prima oara, o sa explicam notiunile si resursele folosite pentru studiu in preliminar. Dupa aceea o sa exploram modul in care datele au fost create dar si prelucrate in capitolul 3. 
In capitolul 4 va fi explicata corectarea automata a lucrarilor produse de studenti, dar si feedbackul generat pentru profesor 
(dacă elevii ar face parte din aceeași clasă, ce informații i-ar putea fi oferite profesorului pentru a îmbunătăți predarea sa?)
In cele din urma, o sa expunem rezulatele studiului in concluzie.